\documentclass{acmconf}
\usepackage{graphicx}
\usepackage{multicol}
\usepackage{amssymb,amsmath,epsfig,alltt}
\sloppy
\usepackage{palatino}
\usepackage{pdftricks}
\begin{psinputs}
  \usepackage{pstricks}
  \usepackage{pst-node}
\end{psinputs}
\usepackage{parskip}
\usepackage{tabularx}
\usepackage{alltt}
\bibliographystyle{alpha}

\title{\textbf{\textsf{
NestedVM: Total Translation of Native Code into Safe Bytecode
}}}
\date{}
\author{\begin{tabular}{@{}c@{}}
        {\em {Brian Alliet}} \\
        {Rochester Institute of Technology}\\
        {\tt brian@ibex.org}
   \end{tabular}\hskip 1in\begin{tabular}{@{}c@{}}
        {\em {Adam Megacz}} \\
        {UC Berkeley Statistical Computing Facility} \\
        {\tt adam@ibex.org}
\end{tabular}}
\begin{document}

\maketitle

\begin{abstract}

We present a new approach to utilizing unsafe legacy code
within safe virtual machines by compiling to MIPS machine code as an
intermediate language.  This approach carries N key benefits over
existing techniques:

\begin{itemize}
\item total coverage of all language features, unlike source translation
\item no build process modifications
\item no post-translation human intervention
\item efficient bytecode
\end{itemize}

We also present NestedVM, a complete system in production use which
implements this technique.  We conclude with quantitative performance
measurements and suggestions for VM acceleration of the resulting
bytecodes.


\end{abstract}

\section{Introduction}

The C programming language \cite{KR} has been in use for over 30
years.  Consequently, there is a huge library of software written in
this language.  Although Java offers substantial benefits \cite{} over
C (and C++), its comparatively young age means that it often lacks
equivalents of many C/C++ libraries.

The typical solution to this dilemma is to use JNI \cite{} or CNI
\cite{} to invoke C code from within a Java VM.  Unfortunately, there
are a number of situations in which this is not an acceptable
solution due to security concerns:

\begin{itemize}

\item Java Applets are not permitted to invoke {\tt
      Runtime.loadLibrary()}

\item Java Servlet containers with a {\tt SecurityManager} will not
      permit loading new JNI libraries.  This configuration is popular
      with {\it shared hosting} providers and corporate intranets
      where a number of different parties contribute individual web
      applications which are run together in a single container.

\item Unlike Java Bytecode, JNI code is susceptible to buffer overflow
      and heap corruption attacks.  This can be a major security
      vulnerability.

\end{itemize}

In addition to security concerns, JNI and CNI carry other
disadvantages:

\begin{itemize}

\item JNI requires the native library to be compiled ahead of time,
      separately, for every architecture on which it will be deployed.
      This is unworkable for situations in which the full set of
      target architectures is not known at deployment time.

\item The increasingly popular J2ME \cite{} platform does not support
      JNI or CNI.

\item JNI often introduces undesirable added complexity to an
      application.

\end{itemize}

The technique we present here is based on using a typical ANSI C
compiler to compile C/C++ code into a MIPS binary, and then using a
tool to translate that binary on an instruction-by-instruction basis
into Java bytecode.

The technique presented here is general; we anticipate that it can be
applied to other secure virtual machines such as Microsoft's .NET
\cite{}, Perl Parrot \cite{}, or Python bytecode \cite{}.

\section{Approaches to Translation}

Techniques for translating unsafe code into VM bytecode generally fall
into four categories:

\begin{itemize}
\item source-to-source translation
\item source-to-binary translation
\item binary-to-source translation
\item binary-to-binary translation
\end{itemize}

\begin{figure}[h]
\begin{pdfpic}
\newlength{\MyLength}
\settowidth{\MyLength}{machine code}
\newcommand{\MyBox}[1]{\makebox[\MyLength]{#1}}
\begin{psmatrix}[colsep=3,rowsep=3]
  [name=s0]\MyBox{unsafe source} & [name=s1]\MyBox{safe source}   \\[0pt]
  [name=b0]\MyBox{machine code}  & [name=b1]\MyBox{safe bytecode} \\
  \psset{nodesep=5pt,arrows=->}
  \ncline{s0}{b0}<{\it gcc}
  \ncline{s0}{s1}\aput{:U}{\it c2java}
  \ncline{s0}{b1}\aput{:U}{\it gcc bytecode backend}
  \ncline{s1}{b1}>{\it javac}
\end{psmatrix}
\end{pdfpic}
\caption{\label{lattice} Conversion Lattice with examples of tools specific to a C/JVM scenario}
\end{figure}

\begin{figure}[h]
\begin{pdfpic}
\newlength{\MyLength}
\settowidth{\MyLength}{machine code}
\newcommand{\MyBox}[1]{\makebox[\MyLength]{#1}}
\begin{psmatrix}[colsep=3,rowsep=3,nrot=:U]
  [name=s0]\MyBox{unsafe source} & [name=s1]\MyBox{safe source}   \\[0pt]
  [name=b0]\MyBox{machine code}  & [name=b1]\MyBox{safe bytecode} \\
  \psset{nodesep=5pt,arrows=->}
  \ncline{s0}{b0}<{\it gcc}
  \ncline{s1}{b1}>{\it javac}
  \ncline{b0}{s1}\naput{\it NestedVM}
  \ncline{b0}{s1}\nbput{\it binary-to-source}
\end{psmatrix}
\end{pdfpic}
\caption{\label{lattice2} Conversion Lattice including NestedVM in {\it source-output} mode}
\end{figure}

\begin{figure}[h]
\begin{pdfpic}
\newlength{\MyLength}
\settowidth{\MyLength}{machine code}
\newcommand{\MyBox}[1]{\makebox[\MyLength]{#1}}
\begin{psmatrix}[colsep=3,rowsep=3,nrot=:U]
  [name=s0]\MyBox{unsafe source} & [name=s1]\MyBox{safe source}   \\[0pt]
  [name=b0]\MyBox{machine code}  & [name=b1]\MyBox{safe bytecode} \\
  \psset{nodesep=5pt,arrows=->}
  \ncline{s0}{b0}<{\it gcc}
  \ncline{s1}{b1}>{\it javac}
  \ncline{b0}{b1}\naput{\it NestedVM}
  \ncline{b0}{b1}\nbput{\it binary-to-binary}
\end{psmatrix}
\end{pdfpic}
\caption{\label{lattice3} Conversion Lattice including NestedVM in {\it bytecode-output} mode}
\end{figure}

A diagram showing these four translation approaches in the context of
running C/C++ code within a Java VM is shown in Figure~\ref{lattice}.

\subsection{Existing Work}
\subsubsection{Source-to-Source Translation}

\begin{itemize}
\item c2java
\item commercial products
\end{itemize}

A number of commercial products and research projects attempt to
translate C++ code to Java code, preserving the mapping of C++ classes
to Java classes.  Unfortunately this is problematic since there is no
way to do pointer arithmetic except within arrays, and even in that
case, arithmetic cannot be done in terms of fractional objects.

Mention gcc backend

Many of these products advise the user to tweak the code which results
from the translation.  Unfortunately, hand-modifying machine-generated
code is generally a bad idea, since this modification cannot be
automated.  This means that every time the origin code changes, the
code generator must be re-run, and the hand modifications must be
performed yet again.  This is an error-prone process.

Furthermore, NestedVM does not attempt to read C code directly.  This
frees it from the complex task of faithfully implementing the ANSI C
standard (or, in the case of non ANSI-C compliant code, some other
interface).  This also saves the user the chore of altering their
build process to accomodate NestedVM.

\section{NestedVM}

NestedVM takes a novel approach; it uses compiled machine code as a
starting point for the translation process.  NestedVM has gone through
two iterations:

\begin{itemize}
\item binary-to-source compilation  (Figure~\ref{lattice2})
\item binary-to-binary compilation  (Figure~\ref{lattice3})
\end{itemize}

\subsection{Translation Process}

Translating a legacy library for use within a JVM proceeds as follows:

\begin{enumerate}

\item Compile the source code to a statically linked binary, targeting
      the MIPS R2000 ISA.

\item Invoke {\tt NestedVM} on the statically linked binary.
      Typically this will involve linking against {\tt libc}, which
      translates system requests (such as {\tt open()}, {\tt read()},
      or {\tt write()}) into appropriate invocations of the MIPS
      {\tt SYSCALL} instruction.

\item (If using binary-to-source translation) compile the resulting
      {\tt .java} code using {\tt jikes} or {\tt javac}.

\item (Optional) compile the resulting bytecode into a {\it safe}
      native binary using {\tt gcj}.

\item From java code, invoke the {\tt run()} method on the generated
      class.  This is equivalent to the {\tt main()} entry point.

\end{enumerate}


\subsection{Why MIPS?}

We chose MIPS as a source format for two primary reasons: the
availability of tools to translate legacy code into MIPS binaries, and
the close similarity between the MIPS ISA and the Java Virtual Machine.

The MIPS architecture has been around for quite some time, and is well
supported by the GNU Compiler Collection, which is capable of
compiling C, C++, Java, Fortran, Pascal (with p2c), and Objective C
into MIPS binaries.

The MIPS R2000 ISA bears a striking similarity to the Java Virtual
Machine:

\begin{itemize}

%\item The original MIPS ISA supports only 32-bit aligned memory loads
%      and stores.  This allows NestedVM to represent memory as a Java
%      {\tt int[]} without introducing additional overhead.
\item Most of the instructions in the original MIPS ISA operate only on
      32-bit aligned memory locations. This allows NestedVM to represent
      memory as a Java {\tt int[]} array without introducing additional 
      overhead.

\item Unlike its predecessor, the R2000 supports 32-bit by 32-bit
      multiply and divide instructions as well as a single and double
      precision floating point unit.  These capabilities map nicely
      onto Java's arithmetic instructions.

\end{itemize}


\subsection{Binary-to-Source Compilation}

The first incarnation of NestedVM was a binary-to-source compiler.
This version reads in a MIPS binary and emits Java source code, which
can be compiled with {\tt javac}, {\tt jikes}, or {\tt gcj}.

This implementation was primarily a first step towards the
binary-to-binary compiler.  Conveniently, generating Java source code
frees NestedVM from having to perform simple constant propagation
optimizations, since most Java compilers already do this.  A recurring
example is the treatment of the {\tt r0} register, which is fixed as
{\tt 0} in the MIPS ISA.

Lacking the ability to generate specially optimized bytecode
sequences, a straightforward mapping of the general purpose hardware
registers to 32 {\tt int} fields was optimal.

\begin{figure*}[t]
\begin{minipage}[c]{7in}%
\begin{multicols}{2}
{\footnotesize\begin{verbatim}
private final static int r0 = 0;
private int r1, r2, r3,...,r30;
private int r31 = 0xdeadbeef;
private int pc = ENTRY_POINT;

public void run() {
    for(;;) {
        switch(pc) {
            case 0x10000:
                r29 = r29 - 32;
            case 0x10004:
                r1 = r4 + r5;
            case 0x10008:
                if(r1 == r6) {
                    /* delay slot */
                    r1 = r1 + 1;
                    pc = 0x10018:
                    continue;
                }
            case 0x1000C:
                r1 = r1 + 1;
            case 0x10010:
                r31 = 0x10018;
                pc = 0x10210;
                continue;
            case 0x10014:
                /* nop */
            case 0x10018:
                pc = r31;
                continue;
            ...
            case 0xdeadbeef:
                System.err.println(``Exited.'');
                System.exit(1);
        }
    }
}
\end{verbatim}}
\vspace{1in}
{\footnotesize\begin{verbatim}
public void run_0x10000() {
    for(;;) {
    switch(pc) {
        case 0x10000:
            ...
        case 0x10004:
            ...
        ...
        case 0x10010:
            r31 = 0x10018;
            pc = 0x10210;
            return;
        ...
    }
    }
}

pubic void run_0x10200() {
    for(;;) {
    switch(pc) {
        case 0x10200:
            ...
        case 0x10204:
            ...
    }
    }
}

public void trampoline() {
    for(;;) {
    switch(pc&0xfffffe00) {
            case 0x10000: run_0x10000(); break;
            case 0x10200: run_0x10200(); break;
            case 0xdeadbe00:
                ...
        }
    }
}
\end{verbatim}}
\end{multicols}
\end{minipage}
\caption{\label{code1} Trampoline transformation necessitated by Java's 64kb method size limit}
\end{figure*}

Unfortunately Java imposes a 64kb limit on the size of the bytecode
for a single method.  This presents a problem for NestedVM, and
necessitates a {\it trampoline transformation}, as shown in
Figure~\ref{code1}.  With this trampoline in place somewhat large
binaries can be handled without much difficulty -- fortunately there
is no corresponding limit on the size of a classfile as a whole.

Another interesting problem that was discovered while creating the
trampoline method was javac and Jikes' inability to properly optimize
switch statements.  The code in Figure~\ref{lookupswitch} is compiled
into a comparatively inefficient {\tt LOOKUPSWITCH}, while the code in
Figure~\ref{tableswitch} is optimized into a {\tt TABLESWITCH}.

\begin{figure}
{\footnotesize\begin{verbatim}
switch(pc&0xffffff00) {
    case 0x00000100: run_100(); break;
    case 0x00000200: run_200(); break;
    case 0x00000300: run_300(); break;
}
\end{verbatim}}
\caption{\label{lookupswitch} Code which {\it is not} optimized into a tableswitch}
\end{figure}

\begin{figure}
{\footnotesize\begin{verbatim}
switch(pc>>>8) {
    case 0x1: run_100();
    case 0x2: run_200();
    case 0x3: run_300();
}
\end{verbatim}}
\caption{\label{tableswitch} Code which {\it is} optimized into a tableswitch}
\end{figure}

Javac isn't smart enough to see the pattern in the case values and
generates very suboptimal bytecode. Manually doing the shifts
convinces javac to emit a tableswitch statement, which is
significantly faster. This change alone nearly doubled the speed of
the compiled binary.

Finding the optimal method size lead to the next big performance
increase.  It was determined through experimentation that the optimal
number of MIPS instructions per method is 128 (considering only power
of two options). Going above or below that lead to performance
decreases. This is most likely due to a combination of two factors.

\begin{itemize}

\item The two levels of switch statements jumps have to pass though -
      The first switch statement jumps go through is the trampoline
      switch statement. This is implemented as a {\tt TABLESWITCH} in JVM
      bytecode so it is very fast. The second level switch statement
      in the individual run\_ methods is implemented as a
      {\tt LOOKUPSWITCH}, which is much slower. Using smaller methods puts
      more burden on the faster {\tt TABLESWITCH} and less on the slower
      {\tt LOOKUPSWITCH}.

\item JIT compilers probably favor smaller methods smaller methods are
      easier to compile and are probably better candidates for JIT
      compilation than larger methods.

\end{itemize}

Put a chart in here

The next big optimization was eliminating unnecessary case
statements. Having case statements before each instruction prevents
JIT compilers from being able to optimize across instruction
boundaries. In order to eliminate unnecessary case statements every
possible address that could be jumped to directly needed to be
identified. The sources for possible jump targets come from 3 places.

\begin{itemize}

\item The .text segment - Every instruction in the text segment is
      scanned for jump targets. Every branch instruction (BEQ, JAL,
      etc) has its destination added to the list of possible branch
      targets. In addition, functions that set the link register have
      theirpc+8 added to the list (the address that would've been put
      to the link register). Finally, combinations of LUI (Load Upper
      Immediate) of ADDIU (Add Immediate Unsigned) are scanned for
      possible addresses in the text segment. This combination of
      instructions is often used to load a 32-bit word into a
      register.

\item The .data segment - When GCC generates switch() statements it
      often uses a jump table stored in the .data
      segment. Unfortunately gcc doesn't identify these jump tables in
      any way. Therefore, the entire .data segment is conservatively
      scanned for possible addresses in the .text segment.
      
\item The symbol table - This is mainly used as a backup. Scanning the
      .text and .data segments should identify any possible jump
      targets but adding every function in the symbol table in the ELF
      binary doesn't hurt. This will also catch functions that are
      never called directly from the MIPS binary (for example,
      functions called with the call() method in the runtime).

\end{itemize}

Eliminating unnecessary case statements provided a 10-25\% speed
increase.

Despite all the above optimizations and workaround an impossible to
workaround hard classfile limit was eventually hit, the constant
pool. The constant pool in classfiles is limited to 65536
entries. Every Integer with a magnitude greater than 32767 requires an
entry in the constant pool. Every time the compiler emits a
jump or branch instruction the PC field is set to the branch target. This
means nearly every branch instruction requires an entry in the
constant pool. Large binaries hit this limit fairly quickly. One
workaround that was employed in the Java source compiler was to
express constants as offsets from a few central values. For example:
``pc = N\_0x00010000 + 0x10'' where N\_0x000100000 is a non-final
field to prevent javac from inlining it. This was sufficient to get
reasonable large binaries to compile. It has a small (approximately
5\%) performance impact on the generated code. It also makes the
generated classfile somewhat larger.  Fortunately, the classfile
compiler eliminates this problem.


\subsection{Binary-to-Binary Translation}

The next step in the evolution of NestedVM was to compile directly to
JVM bytecode eliminating the intermediate javac step. This had several
advantages:

\begin{itemize}
      
\item There are little tricks that can be done in JVM bytecode that
      can't be done in Java source code.

\item Eliminates the time-consuming javac step - Javac takes a long
      time to parse and compile the output from the java source
      compiler.

\item Allows for MIPS binaries to be compiled and loaded into a
      running VM using a class loader. This eliminates the need to
      compile the binaries ahead of time.

\end{itemize}

By generating code at the bytecode level there are many areas where
the compiler can be smarter than javac. Most of the areas where
improvements where made where in the handling of branch instructions
and in taking advantage of the JVM stack to eliminate unnecessary
LOADs and STOREs to local variables.

The first obvious optimization that generating bytecode allows for is the
use of GOTO. Despite the fact that java doesn't have a GOTO keyword a GOTO
bytecode does exist and is used heavily in the code generates by javac.
Unfortunately the java language doesn't provide any way to take advantage of
this. As result of this, jumps within a method were implemented in the
binary-to-source compiler by setting the PC field to the new address and
making a trip back to the initial switch statement.  In the classfile
compiler these jumps are implemented as GOTOs directly to the target
instruction. This saves a costly trip back through the LOOKUPSWITCH
statement and is a huge win for small loops within a method.

Somewhat related to using GOTO is the ability to optimize branch
statements. In the Java source compiler branch statements are
implemented as follows (delay slots are ignored for the purpose of
this example):

{\footnotesize\begin{verbatim}
if(condition) { pc = TARGET; continue; }
\end{verbatim}}

This requires a branch in the JVM regardless of whether the MIPS
branch is actually taken. If condition is false the JVM has to jump
over the code to set the PC and go back to the switch block. If
condition is true the JVM as to jump to the switch block. By
generating bytecode directly we can make the target of the JVM branch
statement the actual bytecode of the final destination. In the case
where the branch isn't taken the JVM doesn't need to branch at all.

A side affect of the above two optimizations is a solution to the
excess constant pool entries problem. When jumps are implemented as
GOTOs and direct branches to the target the PC field doesn't need to
be set. This eliminates many of the constant pool entries the java
source compiler requires. The limit is still there however, and given
a large enough binary it will still be reached.

Delay slots are another area where things are done somewhat
inefficiently in the Java source compiler. In order to take advantage
of instructions already in the pipeline MIPS cpu have a ``delay
slot''. That is, an instruction after a branch or jump instruction that
is executed regardless of whether the branch is taken. This is done
because by the time the branch or jump instruction is finished being
processes the next instruction is already ready to be executed and it
is wasteful to discard it. (However, newer MIPS CPUs have pipelines
that are much larger than early MIPS CPUs so they have to discard many
instructions anyway.) As a result of this the instruction in the delay
slot is actually executed BEFORE the branch is taken. To make things
even more difficult, values from the register file are loaded BEFORE
the delay slot is executed.  Here is a small piece of MIPS assembly:

{\footnotesize\begin{verbatim}
ADDIU r2,r0,-1
BLTZ r2, target
ADDIU r2,r2,10
...
:target
\end{verbatim}}

This piece of code is executed as follows

\begin{enumerate}

\item r2 is set to -1

\item r2 is loaded from the register file by the BLTEZ instruction
      
\item 10 is added to r2 by the ADDIU instruction

\item The branch is taken because at the time the BLTZ instruction was
      executed r2 was -1, but r2 is now 9 (-1 + 10)

\end{enumerate}

There is a very elegent solution to this problem when using JVM
bytecode. When a branch instruction is encountered the registers
needed for the comparison are pushed onto the stack to prepare for the
JVM branch instruction. Then, AFTER the values are on the stack the
delay slot is emitted, and then finally the actual JVM branch
instruction. Because the values were pushed to the stack before the
delay slot was executed any changes the delay slot made to the
registers are not visible to the branch bytecode. This allows delay
slots to be used with no performance penalty or size penalty.

One final advantage that generating bytecode directly allows is
smaller more compact bytecode. All the optimization above lead to
smaller bytecode as a side effect. There are also a few other areas
where the generated bytecode can be optimized for size with more
knowledge of the program as a whole.

When encountering the following switch block both javac and Jikes
generate redundant bytecode.

{\footnotesize\begin{verbatim}
switch(pc>>>8) {
    case 0x1: run_1(); break;
    case 0x2: run_2(); break
    ...
    case 0x100: run_100(); break;
}
\end{verbatim}}

The first bytecode in each case arm in the switch statement is ALOAD\_0 to
prepare for a invoke special call. By simple moving this outside the switch
statement each case arm was reduced in size by one instruction. Similar
optimizations were also done in other parts of the compiler.


\section{Interfacing with Java Code}

Java source code can create a copy of the translated binary by
instantiating the corresponding class, which extends {\tt Runtime}.
Invoking the {\tt main()} method on this class is equivalent to
calling the {\tt main()} function within the binary; the {\tt String}
arguments to this function are copied into the binary's memory space
and made available as {\tt **argv} and {\tt argc}.

The translated binary communicates with the rest of the VM by
executing MIPS {\tt SYSCALL} instructions, which are translated into
invocations of the {\tt syscall()} method.  This calls back to the
native Java world, which can manipulate the binary's environment by
reading and writing to its memory space, checking its exit status,
pausing the VM, and restarting the VM.


\subsection{Virtualization}

The {\tt Runtime} class implements the majority of the standard {\tt
libc} syscalls, providing a complete interface to the filesystem,
network socket library, time of day, (Brian: what else goes here?).

\begin{itemize}

\item ability to provide the same interface to CNI code and
      NestedVMified code
      
\item security advantages (chroot the {\tt fork()}ed process)

\end{itemize}


\section{Quantitative Performance}

\subsection{Charts}

(Note that none of these libraries have pure-Java equivalents.)

\begin{itemize}
\item libjpeg
\item libfreetype
\item libmspack
\end{itemize}


\subsection{Optimizations}

Brian, can you write something to go here?  Just mention which
optimizations helped and which ones hurt.

\begin{itemize}
\item {\tt trampoline}
\item {\tt optimal method size}
\item {\tt -msingle-float}
\item {\tt -mmemcpy}
\item {\tt fastmem}
\item {\tt local vars for registers (useless)}
\item {\tt -fno-rename-registers}
\item {\tt -ffast-math}
\item {\tt -fno-trapping-math}
\item {\tt -fsingle-precision-constant}
\item {\tt -mfused-madd}
\item {\tt -freg-struct-return}
\item {\tt -freduce-all-givs}
\item {\tt -fno-peephole}
\item {\tt -fno-peephole2}
\item {\tt -fmove-all-movables}
\item {\tt -fno-sched-spec-load}
\item {\tt -fno-sched-spec}
\item {\tt -fno-schedule-insns}
\item {\tt -fno-schedule-insns2}
\item {\tt -fno-delayed-branch}
\item {\tt -fno-function-cse}
\item {\tt -ffunction-sections}
\item {\tt -fdata-sections}
\item {\tt array bounds checking}
\item {\tt -falign-functions=n}
\item {\tt -falign-labels=n}
\item {\tt -falign-loops=n}
\item {\tt -falign-jumps=n}
\item {\tt -fno-function-cse}
\end{itemize}

\section{Future Directions}

World domination.

\section{Conclusion}

We rock the hizzouse.

\section{References}

Yer mom.

\section{stuff}
\begin{onecolumn}
{\footnotesize\begin{verbatim}

libjpeg (render thebride_1280.jpg)
Native -  0.235s
JavaSource - 1.86s
ClassFile - 1.37s

freetype (rendering characters 32-127 of Comic.TTF at sizes from 8 to
48 incrementing by 4)
Native - 0.201s
JavaSource - 2.02s
ClassFile - 1.46s

                                          libjpeg  libmspack libfreetype
Interpreted MIPS Binary                   22.2      12.9      21.4
Compled MIPS Binary (fastest options)     3.39      2.23      4.31
Native -O3                                0.235    0.084     0.201

Compled - with all case statements        3.50      2.30      4.99
Compiled - with pruned case statement     3.39      2.23      4.31

Compiled - 512 instructions/method        62.7      27.7      56.9
Compiled - 256 instructions/method        3.54      2.55      4.43
Compiled - 128 instructions/method        3.39      2.23      4.31
Compiled - 64 instructions/method         3.56      2.31      4.40
Compiled - 32 instruction/method          3.71      2.46      4.64

Compild MIPS Binary (Server VM)           3.21      2.00      4.54
Compiled MIPS Binary (Client VM)          3.39      2.23      4.31

All times are measured in seconds. These were all run on a dual 1ghz G4
running OS X 10.3.1 with Apple's latest VM (JDK 1.4.1_01-27). Each test
was run 8 times within a single VM. The highest and lowest times were
removed and the remaining 6 were averaged. In each case only the first
run differed significantly from the rest.

The libjpeg test consisted of decoding a 1280x1024 jpeg
(thebride_1280.jpg) and writing a tga. The mspack test consisted of
extracting all members from arial32.exe, comic32.exe, times32.exe, and
verdan32.exe. The freetype test consisted of rendering characters
32-127 of Comic.TTF at sizes from 8 to 48 incrementing by 4. (That is
about 950 individual glyphs).

I can provide you with the source for any of these test if you'd like.

-Brian
\end{verbatim}}
\end{onecolumn}
\end{document}

